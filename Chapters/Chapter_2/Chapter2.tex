% Chapter Template

\chapter{Literature Review}\doublespacing % Main chapter title

\label{Chapter2} % Change X to a consecutive number; for referencing this chapter elsewhere, use \ref{ChapterX}

\lhead{Chapter II. \emph{Literature Review}} % Change X to a consecutive number; this is for the header on each page - perhaps a shortened title

% -----------------------------
%  Introduction
% -----------------------------

\section{Introduction}
\noindent The goal of this literature review is to investigate the state of current research on sustainable supply chain management practices and the use of blockchain and machine learning in this field. Recent years have seen a substantial increase in interest in sustainable supply chain management because of its potential to reduce environmental impact, increase social responsibility, and boost economic performance.
\par In order to perform this assessment, we looked at a number of academic publications and articles that explicitly look into supply chain sustainability, blockchain integration, and machine learning. The selected research addresses various facets of these subjects and offers insightful information on their consequences and efficacy.

\section{Related Work}
\begin{itemize}
    \item Tian proposed a food supply chain traceability \cite{bib1} 
    based on Hazard Analysis and Critical Control Points 
    (HACCP) using blockchain and Internet of Things (IoT). 
    Tian also discussed the advantages and disadvantages of 
    RFID and blockchain for agriculture food supply chain 
    traceability in a previous work. Caro et al. presented 
    AgriBlockIoT, a blockchain-based traceability solution 
    that integrates data from IoT devices along the value 
    chain, with implementation comparisons using both 
    Ethereum and Hyperledger. Tse et al. discussed at a high 
    abstract level how blockchain technology can be applied 
    to food supply chains and compared it with traditional 
    solutions. They also highlighted key aspects related to 
    security, integrity, and trust. Lin et al. reviewed 
    blockchain concepts for Agri-ICT systems and presented 
    a model ICT system for agriculture using blockchain 
    technology.
    
    \item Tripoli and Schmidhuber discussed the application of \cite{bib2}
    distributed ledger technologies (DLT) and smart 
    contracts for increasing efficiency and providing 
    traceability in agriculture. The authors identified 
    technical challenges and barriers to adoption and 
    concluded that DLTs have significant potential in 
    achieving sustainable development goals. Mao and 
    Dianhui presented a blockchain-based credit evaluation 
    system via smart contracts for efficient management in 
    food supply chains. Galves et al. reviewed challenges and 
    potential uses of blockchain for assuring traceability and 
    authenticity in food supply chains. Mao et al. proposed a 
    consortium blockchain approach to an efficient food 
    trading system and validated it using a case study in 
    Shandong province, China. Lucena et al. presented an 
    approach for grain quality measurement using blockchain 
    and smart contracts and implemented a solution for a real life case that resulted in 15\% added valuation for 
    genetically modified (GM)-free soy grain exports from 
    Brazil. Chinaka studied how implementing a blockchain based solution can facilitate value transfer in small-scale 
    agriculture in Africa by translating farmer's assets such as 
    livestock, farmlands, and produce . Schneider designed a 
    prototype blockchain system to enhance transparency and 
    automate processes in the agricultural sector. Holmberg 
    and Aquist studied the challenges in implementing a 
    blockchain-based traceability solution in the dairy 
    industry.
    
    \item Agridigital's use of Ethereum to ease the exchange of \cite{bib3}
    wheat in Australia is one recent example of a blockchain 
    trial application in the food and agriculture supply 
    chains, as are Walmart's food traceability pilots with 
    IBM Hyperledger. Additionally, the first-ever 
    blockchain-based commodities trade of US soybeans to 
    China's Shangong Bohi Industry was successfully 
    conducted by Louis Dreyfus, a significant commodity 
    trader.
    
    \item The adoption of blockchain technology for improved \cite{bib4}
    information security, transparency, and authentication in 
    different parts of food and agricultural supply chains is 
    on the rise, as is shown from these connected works. 
    While lacking particular implementation frameworks or 
    methodologies, a sizable amount of the literature 
    continues to concentrate on the conceptual application of 
    blockchain in agricultural supply chains. 
    Our project contributes to the growing literature on 
    blockchain applications by proposing an efficient, 
    reliable, secure, and decentralized trace and track solution 
    for supply chains. We leverage the power of 
    blockchain technology, machine learning, and 
    Hyperledger Fabric smart contracts to create a 
    comprehensive system. Our work includes detailed 
    information on the system's architecture, metadata, 
    sequence diagrams and interactions, applied to various 
    scenarios involving agricultural supply chains with 
    multiple stakeholders. This addition to the existing 
    literature on blockchain applications in agriculture will 
    provide valuable insights and contribute to the 
    advancement of the field.
  \end{itemize}


% \section{Conclusion}
\noindent The literature demonstrates an increasing adoption of blockchain technology in food and agriculture supply chains to enhance information security, transparency, and authentication. While some works focus on conceptual applications, our project contributes by proposing an efficient, secure, decentralized trace and track solution for products in supply chains. Leveraging blockchain technology, machine learning, and Hyperledger Fabric smart contracts, our comprehensive system includes architecture, metadata, sequence diagrams, and stakeholder interactions. This contribution adds valuable insights to the existing literature on blockchain applications in agriculture and advances the field as a whole.