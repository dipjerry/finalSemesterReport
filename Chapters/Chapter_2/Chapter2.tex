% Chapter Template

\chapter{Literature Review}\doublespacing % Main chapter title

\label{Chapter2} % Change X to a consecutive number; for referencing this chapter elsewhere, use \ref{ChapterX}

\lhead{Chapter II. \emph{Literature Review}} % Change X to a consecutive number; this is for the header on each page - perhaps a shortened title

% -----------------------------
%  Introduction
% -----------------------------

\section{Introduction}
\lipsum[1-2]

% -----------------------------
%  Factors Influencing
% -----------------------------

\section{Factors Influencing \ldots \ldots}
Tortor sed accumsan bibendum, erat ligula aliquet magna, vitae ornare odio metus a mi. Morbi ac orci et nisl hendrerit mollis. Suspendisse ut massa. Cras nec ante. Pellentesque a nulla. Cum sociis natoque penatibus et magnis dis parturient montes, nascetur ridiculus mus. Aliquam tincidunt urna. Nulla ullamcorper vestibulum turpis. Pellentesque cursus luctus mauris Figure \ref{fig:figure2_1}. Nulla ullamcorper vestibulum turpis. Pellentesque cursus luctus mauris:

% adjustbox is used to limit the figure inside page
% -- means normal arrow
%  -| horizontal then vertical arrow
%  |- vertical then horizontal arrow


\begin{figure}

    \begin{center}
        \begin{adjustbox}{max height=\textheight, center, width = 0.5\textwidth}
            \begin{tikzpicture}[node distance=2cm]
                % Layer 1
                \node (1) [startstop1] {\textbf{Step 1}};
    
                % Layer 2
                \node (2a) [startstop1, left of = 1, xshift = -7cm, yshift = -2cm] {Step 1.1};
                \node (2b) [startstop1, below of = 1, yshift = -4cm, xshift = -2.7cm] {Step 1.2};
                \node (2c) [startstop1, right of = 1, xshift = 1.5cm, yshift = -10cm] {Step 1.3};
                \node (2d) [startstop1, right of = 1, xshift = 6.5cm, yshift = -14cm] {Step 1.4};
    
                % Arrows Layer 1  to Layer 2
                \draw [arrow] (1) -| (2a);
                \draw [arrow] (1) -| (2b);
                \draw [arrow] (1) -| (2c);
                 \draw [arrow] (1) -| (2d);
                
                % Layer 3a
                \node (2a_3b) [startstop1, below of = 2a] {Step 1.1.2};
                \node (2a_3a) [startstop1, left of = 2a_3b, xshift = -2cm] {Step 1.1.1};
                \node (2a_3c) [startstop1, right of = 2a_3b, xshift = 2cm] {Step 1.1.3};
    
                % Arrows Layer 2a  to Layer 3a
                \draw [arrow] (2a) -| (2a_3a);
                \draw [arrow] (2a) -- (2a_3b);
                \draw [arrow] (2a) -| (2a_3c);
                
                
                % Layer 3b
                \node (2b_3b) [startstop1, below of = 2b] {Step 1.2.2};
                \node (2b_3a) [startstop1, left of = 2b_3b, xshift = -2cm] {Step 1.2.1};
                \node (2b_3c) [startstop1, right of = 2b_3b, xshift = 2cm] {Step 1.2.3};

                % Arrows Layer 2b to Layer 3b
                \draw [arrow] (2b) -- (2b_3b);
                \draw [arrow] (2b) -| (2b_3a);
                \draw [arrow] (2b) -| (2b_3c);

                % Layer 3c
                % \node (2c_3b) [startstop1,  below of = 2c] {Diversity};
                \node (2c_3a) [startstop1, left of = 2c, xshift = -1cm, yshift = -2cm]  {Step 1.3.1};
                \node (2c_3b) [startstop1, right of = 2c, xshift = 1cm, yshift = -2cm] {Step 1.3.2};
    
                % Arrows Layer 2c to Layer 3c
                % \draw [arrow] (2c) -- (2c_3b);
                \draw [arrow] (2c) -| (2c_3a);
                \draw [arrow] (2c) -| (2c_3b);

                 % Layer 3d
                \node (2d_3b) [startstop1, left of = 2d, xshift = -3.5cm, yshift = -2cm] {Step 1.4.1};
                \node (2d_3a) [startstop1, left of = 2d, xshift = 0cm, yshift = -2cm] {Step 1.4.2};
                \node (2d_3c) [startstop1, right of = 2d, xshift = 0cm, yshift = -2cm] {Step 1.4.3};
                \node (2d_3d) [startstop1, right of = 2d, xshift = 3.5cm, yshift = -2cm] {Step 1.4.4};

                % Arrows Layer 2d to Layer 3d
                \draw [arrow] (2d) -| (2d_3b);
                \draw [arrow] (2d) -| (2d_3a);
                \draw [arrow] (2d) -| (2d_3c);
                \draw [arrow] (2d) -| (2d_3d);
                
            \end{tikzpicture}
        \end{adjustbox}
    \end{center}
    \caption{Factors influencing \ldots \ldots \ldots \ldots \ldots \ldots}
    \label{fig:figure2_1}
\end{figure}


\subsection{SubFactors}
\lipsum[1-2]

\subsubsection{SubSubFactors}
Nulla ullamcorper vestibulum turpis. Pellentesque cursus luctus mauris Table \ref{tab:table2_1}.
\lipsum[3-5]

% Table generated by Excel2LaTeX from sheet 'Sheet1'
\begin{table}[htbp]
  \centering
  \caption{Studies related to \ldots \ldots}
  \resizebox{\textwidth}{!}{%
    \begin{tabular}{llllll}
    
    \toprule
    \textbf{\makecell[l]{Authors\\ and\\ Country}} & \textbf{\makecell[l]{Population}} & \textbf{\makecell[l]{Study\\ Type}} & \textbf{\makecell[l]{Sample\\ Size}} & \textbf{Variable used} & \textbf{\makecell[l]{Model/\\method\\ used}} \\ \midrule
  
    \makecell[l]{Author 1,\\ India} & \makecell[l]{Young\\ pedestrians} & \makecell[l]{O} & \makecell[l]{1033} & \makecell[l]{Var 1, Var 2, Var 3, Var 4,\\ Var 5, Var 6, and Var 7}
    & EDA\\ \midrule
    
    
    \makecell[l]{Author 2,\\ Australia} & \makecell[l]{6th grade\\ students} & \makecell[l]{Q} & \makecell[l]{405} & \makecell[l]{Var 1, Var 2, Var 3, Var 4,\\ Var 5, Var 6, and Var 7}
    & Regression\\ \bottomrule
    
    \end{tabular}
    }

    \begin{flushleft}
    \vspace{0.2cm}
        {\scriptsize \textbf{Note:} E: Experimental, O: Observational survey, Q: Questionnaire survey, R: Review, V: Video graphic survey.\\
        \hspace{0.85cm} EDA: Exploratory Data Analysis}
    \end{flushleft}

  \label{tab:table2_1}%
\end{table}%


\subsubsection{SubSubFactors}
\lipsum[5-7]


\subsubsection{SubSubFactors}
\lipsum[8-9]

\subsection{SubFactors}
\lipsum[1]


\subsubsection{SubSubFactors}
Speeding is a well-documented cause of crashes involving pedestrians, particularly children. To decrease the number of collisions involving school-going children, speed limits have been implemented in school zones worldwide since the 90s. However, despite these efforts, the number of fatal crashes caused by speeding has risen and continues to do so. This suggests that current strategies and campaigns are not effectively communicating the dangers of speeding to drivers. Additionally, there is a lack of data on the daily occurrences of speeding in these zones, except for enforcement efforts carried out by law enforcement. Moreover, variations in school zone treatments across different countries do not fully explain why drivers choose to speed in these areas.

\lipsum[7-8]

\subsubsection{SubSubFactors}

\lipsum

\subsection{SubFactors}
Lorem ipsum dolor sit amet, consectetuer adipiscing elit. Ut purus elit, vestibulum ut, placerat ac, adipiscing vitae, felis. Curabitur dictum gravida mauris. Nam arcu libero, nonummy eget, consectetuer id, vulputate a, magna. Donec vehicula augue eu neque Figure \ref{fig:figure2_2}.

\begin{figure}[htbp]
\centering
\includegraphics[width=0.4\textwidth]{Chapters/Chapter_2/figures/image2_2.PNG}
\caption{Plot with a single image (Image by pencil parker from Pixabay)}
\label{fig:figure2_2}
\end{figure}

\lipsum[1-3]


% -----------------------------
%  Research Gaps from the Literature Review
% -----------------------------

\section{Research Gaps from the Literature Review}
Fusce mauris. Vestibulum luctus nibh at lectus. Sed bibendum, nulla a faucibus semper, leo velit ultricies tellus:

\begin{itemize}

    \item Fusce mauris. Vestibulum luctus nibh at lectus. Sed bibendum, nulla a faucibus semper, leo velit ultricies tellus, ac venenatis arcu wisi vel nisl. Vestibulum diam.

    \item Fusce mauris. Vestibulum luctus nibh at lectus. Sed bibendum, nulla a faucibus semper, leo velit ultricies tellus, ac venenatis arcu wisi vel nisl. Vestibulum diam.
    
    \item Fusce mauris. Vestibulum luctus nibh at lectus. Sed bibendum, nulla a faucibus semper, leo velit ultricies tellus, ac venenatis arcu wisi vel nisl. Vestibulum diam.
    
    \item Fusce mauris. Vestibulum luctus nibh at lectus. Sed bibendum, nulla a faucibus semper, leo velit ultricies tellus, ac venenatis arcu wisi vel nisl. Vestibulum diam.
    
    \item Fusce mauris. Vestibulum luctus nibh at lectus. Sed bibendum, nulla a faucibus semper, leo velit ultricies tellus, ac venenatis arcu wisi vel nisl. Vestibulum diam.

\end{itemize}

\section{Summary}
\lipsum[1]