% Chapter Template

\chapter{Methodology}\doublespacing % Main chapter title

\label{Chapter4} % Change X to a consecutive number; for referencing this chapter elsewhere, use \ref{ChapterX}

\lhead{Chapter IV. \emph{Methodology}} % Change X to a consecutive number; this is for the header on each page - perhaps a shortened title


% --------------------------------
% Study Summary
% --------------------------------

\section{Introduction}
\lipsum[1-4]

% ----------------------------
% Methodology
% ----------------------------

\section{Objective}
\noindent In our project, we developed a supply chain management application based on blockchain and machine learning using a strict and iterative methodological approach. Our objective was to create a scalable and adaptable system that addresses the challenges and constraints associated with traditional supply chain management methods. By combining the advantages of blockchain technology with machine learning algorithms, we hoped to change supply networks.
\section{Data Collection Method}
\noindent To ensure the adequacy of data for our study, we adopted a dual approach to data collection, utilizing both primary and secondary methods. Our primary data collection involved conducting interviews and surveys with various individuals closely associated with the supply chain domain, including experts, industry practitioners, and potential end-users. These interactions provided us with valuable firsthand information regarding their specific needs, challenges, and expectations related to supply chain management. By engaging with these stakeholders directly, we gained a deep understanding of the intricacies and complexities of the field.
Furthermore, we complemented our primary data with secondary data from diverse sources. We extensively reviewed research papers, case studies, and industry reports to obtain additional insights into the latest trends and best practices in supply chain management. This secondary data enabled us to enrich our understanding of the subject matter, validate our primary findings, and acquire a broader perspective on the current state of the industry.
By combining primary data obtained through interviews and surveys with secondary data derived from research papers, case studies, and industry reports, we ensured a comprehensive and well-informed approach to our study. This dual data collection methodology allowed us to capture a wide range of perspectives, validate our findings, and present a thorough analysis of the supply chain landscape.

\section{Conslusion}
\noindent The decision to integrate blockchain and machine learning into our supply chain management app was driven by several key factors. One of the main drivers was the recognition of the need for improved transparency, traceability, and security in supply chain operations. By incorporating blockchain technology, we aimed to create a decentralized and immutable ledger that would ensure data integrity and enhance trust among all stakeholders involved in the supply chain.
\par Blockchain technology provides a transparent and tamper-resistant platform where all transactions and information are securely recorded. This enables real-time visibility into the movement of goods, from the point of origin to the final destination. With the use of smart contracts, we can automate and enforce contractual agreements, ensuring compliance and reducing the risk of fraud or error.    
