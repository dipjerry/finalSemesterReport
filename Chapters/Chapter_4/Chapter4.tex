% Chapter Template

\chapter{Methodology}\doublespacing % Main chapter title

\label{Chapter4} % Change X to a consecutive number; for referencing this chapter elsewhere, use \ref{ChapterX}

\lhead{Chapter IV. \emph{Methodology}} % Change X to a consecutive number; this is for the header on each page - perhaps a shortened title


% --------------------------------
% Study Summary
% --------------------------------

\section{Introduction}
\noindent{The methodology employed in implementing the proposed strategy for enhancing trust and traceability in supply chains through blockchain and machine learning involves a comprehensive approach that integrates various components to address the challenges and achieve the desired outcomes in the agriculture and food sectors.

The methodology begins with a thorough analysis of the existing supply chain processes, identifying the key pain points, inefficiencies, and areas where transparency and traceability are lacking. This analysis helps in understanding the specific needs and requirements of the supply chains, laying the foundation for the implementation of blockchain and machine learning technologies.

The first step in the methodology is the integration of blockchain technology into the supply chains. This involves designing and developing a secure and decentralized platform that can track and record each transaction and activity along the supply chain. The platform ensures immutability, transparency, and security of the captured data, reducing the need for intermediaries and accelerating the overall process. Technical challenges related to scalability and interoperability are addressed during this phase, ensuring the seamless integration of blockchain into the existing infrastructure.

Next, machine learning algorithms are incorporated into the system to leverage the vast amount of data stored on the blockchain. These algorithms are designed to analyze the data and provide intelligent analysis, forecasts, and recommendations for improving supply chain operations. For instance, weather data, soil conditions, and crop health information can be analyzed to optimize irrigation, fertilization, and pest management, leading to increased crop yields and sustainable farming practices. Additionally, data on product quality and transportation routes can be analyzed to identify potential issues and implement preventive actions, ensuring the delivery of high-quality products to consumers.

Data governance and privacy measures are implemented as an integral part of the methodology to address concerns related to data security and privacy. Robust protocols are established to ensure the confidentiality and integrity of the data stored on the blockchain, while also complying with relevant regulations and standards. This instills trust among stakeholders and facilitates the adoption and acceptance of the technology.

Interoperability standards are defined and implemented to enable seamless data exchange and integration between different stakeholders along the supply chain. This ensures that information flows smoothly, enhancing collaboration and coordination among farmers, producers, importers, retailers, regulatory agencies, and certifying bodies.

Stakeholder engagement tactics form a crucial part of the methodology. It involves educating and training the various stakeholders on the benefits and functionalities of the blockchain and machine learning technologies. This helps in overcoming resistance to change and promoting widespread adoption and integration of the new system. Collaboration among stakeholders is fostered, encouraging transparency, sharing of information, and continuous improvement of the supply chains.

Throughout the implementation process, regular monitoring and evaluation are conducted to assess the effectiveness of the strategy. Key performance indicators are defined to measure the improvements in efficiency, sustainability, transparency, and customer satisfaction. Feedback from stakeholders is collected and analyzed to identify any further refinements or adjustments needed in the system.

In conclusion, the methodology for enhancing trust and traceability in supply chains through blockchain and machine learning encompasses a comprehensive approach that integrates various components. It involves the integration of blockchain technology, implementation of machine learning algorithms, establishment of data governance and privacy measures, adoption of interoperability standards, and engagement of stakeholders. By following this methodology, the goal of transforming agriculture and food supply chains into more transparent, sustainable, and efficient systems can be achieved}

% ----------------------------
% Methodology
% ----------------------------

\section{Objective}
\noindent In our project, we developed a supply chain management application based on blockchain and machine learning using a strict and iterative methodological approach. Our objective was to create a scalable and adaptable system that addresses the challenges and constraints associated with traditional supply chain management methods. By combining the advantages of blockchain technology with machine learning algorithms, we hoped to change supply networks.
\section{Data Collection Method}
\noindent To ensure the adequacy of data for our study, we adopted a dual approach to data collection, utilizing both primary and secondary methods. Our primary data collection involved conducting interviews and surveys with various individuals closely associated with the supply chain domain, including experts, industry practitioners, and potential end-users. These interactions provided us with valuable firsthand information regarding their specific needs, challenges, and expectations related to supply chain management. By engaging with these stakeholders directly, we gained a deep understanding of the intricacies and complexities of the field.
Furthermore, we complemented our primary data with secondary data from diverse sources. We extensively reviewed research papers, case studies, and industry reports to obtain additional insights into the latest trends and best practices in supply chain management. This secondary data enabled us to enrich our understanding of the subject matter, validate our primary findings, and acquire a broader perspective on the current state of the industry.
By combining primary data obtained through interviews and surveys with secondary data derived from research papers, case studies, and industry reports, we ensured a comprehensive and well-informed approach to our study. This dual data collection methodology allowed us to capture a wide range of perspectives, validate our findings, and present a thorough analysis of the supply chain landscape.\cite{bib5}In system evolution criteria Immutable Records ensuring data integrity by comparing recorded data with traceability results.Transparency enabling an open and auditable supply chain process.User Access Control implementing secure and flexible user access based on their roles.Interoperability allowing system access through different devices.Scalability handling large-scale transactions concurrently.Cost-effectiveness maintaining reasonable operational costs, including transaction fees. User Adoption providing a user-friendly experience for system access and gathering user feedback.

% Table generated by Excel2LaTeX from sheet 'Sheet1'
\begin{table}[htbp]
  \centering
  \caption{System Evaluation  Criteria \ldots}
  \resizebox{\textwidth}{!}{%
    \begin{tabular}{lllllllll}
    
    \toprule
    \textbf{\makecell[l]{Criteria}} & \textbf{\makecell[l]{Description}} & \textbf{\makecell[l]{Measurement}} \\ \midrule
  
    \makecell[l]{Immutable Records} & \makecell[l]{Guarantee data records integrity} & \makecell[l]{Compare data from the recording point with final information in traceability results.} \\ \midrule

    \makecell[l]{Transparency} & \makecell[l]{Provide an open and auditable process} & \makecell[l]{Review the transparency of the supply chain stream.} \\ \midrule

    \makecell[l]{User Access Control} & \makecell[l]{Ensure the system has a secure and flexible user access\\ control mechanism} & \makecell[l]{Check whether the user has appropriate access to record data according to their respective role\\in the supply chain location.} \\ \midrule

    \makecell[l]{Interoperability} & \makecell[l]{Guarantee the system can be accessed by different devices} & \makecell[l]{Check whether the user can read and write to the system by using different devices.} \\ \midrule

    \makecell[l]{Scalability} & \makecell[l]{Ability to handle large scale transaction} & \makecell[l]{Measure how many transactions can be handled concurrently by the system.} \\ \midrule

    \makecell[l]{Cost-effectiveness} & \makecell[l]{Assure the operational cost, such as transaction fees are\\reasonable} & \makecell[l]{Calculate the average cost of transaction.} \\ \midrule

    \makecell[l]{User Adoption} & \makecell[l]{Provide easiness for the user to access the system.} & \makecell[l]{Survey/Questionnaire about user experience in interacting with traceability system.} \\ \bottomrule
    
    \end{tabular}
    }
    % \begin{flushleft}
    % \vspace{0.2cm}
    %     {\scriptsize \textbf{Note:} E: Experimental, O: Observational survey, Q: Questionnaire survey, R: Review, V: Video graphic survey.\\
    %     \hspace{0.85cm} EDA: Exploratory Data Analysis}
    % \end{flushleft}

  \label{tab:table4_1}%
\end{table}%

% \section{Conslusion}
\noindent The decision to integrate blockchain and machine learning into our supply chain management app was driven by several key factors. One of the main drivers was the recognition of the need for improved transparency, traceability, and security in supply chain operations. By incorporating blockchain technology, we aimed to create a decentralized and immutable ledger that would ensure data integrity and enhance trust among all stakeholders involved in the supply chain.
\par Blockchain technology provides a transparent and tamper-resistant platform where all transactions and information are securely recorded. This enables real-time visibility into the movement of goods, from the point of origin to the final destination. With the use of smart contracts, we can automate and enforce contractual agreements, ensuring compliance and reducing the risk of fraud or error. Some popular blockchain platform like Bitcoin uses Proof of Work as the consensus algorithm and has limited smart contract support. It has low scalability and operates on a public network. Development is done using C++.

Ethereum is transitioning from Proof of Work to Proof of Stake as the consensus algorithm. It supports smart contracts written in Solidity and has moderate scalability. Ethereum operates on a public network and supports tokenization through standards like ERC-20 and ERC-721.

Binance Smart Chain uses Proof of Stake as the consensus algorithm and supports smart contracts written in Solidity. It offers high scalability and operates on a public network. Tokenization is supported through the BEP-20 standard.

Cardano utilizes Proof of Stake as the consensus algorithm and supports smart contracts written in Plutus and Marlowe. Cardano offers high scalability and operates on a public network. Development is done using Plutus and Haskell, and tokenization is supported through the ADA cryptocurrency.

Polkadot employs the Nominated Proof of Stake consensus algorithm and supports smart contracts written in Ink!. Polkadot offers high scalability on a public network and development can be done using Ink! and Rust. Tokenization is supported through the DOT cryptocurrency.

Solana utilizes Proof of History as the consensus algorithm and supports smart contracts written in Rust, C, and C++. Solana offers high scalability on a public network and supports tokenization using the ERC-20 standard.

Hyperledger Fabric uses the Practical Byzantine Fault Tolerance consensus algorithm and supports smart contracts written in Chaincode. Hyperledger Fabric offers high scalability and operates on a private network. Development is done using Go, Java, and Node.js.

Corda has a pluggable consensus algorithm and supports smart contracts written in Corda Contract. Corda offers moderate scalability and operates on a private network. Development is done using Kotlin and Java, and tokenization support is not applicable.    
