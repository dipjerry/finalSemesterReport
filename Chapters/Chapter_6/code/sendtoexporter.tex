\begin{lstlisting}[language=Python , caption=send to next stackholder]
input pId,eId:
if pId or eId exists in the network:
User <-Get(eId);
if user.type != `exporter`':
product=Get(eId);
txTimestamp=Get(timestamp);			 
invoice=[name:product.name,price:product.product.price,quantity:product.product.quantity]
product.exporter.id= eId;
product(availablefor==`importer`',status==`processing`',exporter status==`processing`');
return success;       
\end{lstlisting}
The ``sendToNextStackholder" function facilitates the process of sending a product to an next stackholder in the network. It takes the product ID and stackholder ID as inputs. The function performs several verifications to ensure the validity of the inputs. It checks if the product and exporter exist in the system and verifies that the exporter has the appropriate user type. Once the verifications are successful, the function retrieves the product details and validates if the product has already been sent to an the stackholder. If the product is available, the function proceeds to create an invoice and update the necessary product and information. The stackholder ID, export date, product owner, available status, and invoice details are updated accordingly. The updated product object is serialized and stored back into the system. This process ensures that the product is successfully sent to the nextstackholder and marked as ``Processing" for further export-import related activities.