% Chapter Template

\chapter{Implementation}\doublespacing % Main chapter title

\label{Chapter6} % Change X to a consecutive number; for referencing this chapter elsewhere, use \ref{ChapterX}

\lhead{Chapter VI. \emph{Implementation}} % Change X to a consecutive number; this is for the header on each page - perhaps a shortened title

% ----------------------------
% Introduction
% ----------------------------
\section{Introduction}
During the implementation phase, we, as a team, focused on translating the design and requirements into a functional and robust system. This phase involved the actual development and deployment of the supply chain management solution using blockchain technology. We encountered various challenges along the way, but through collaboration and perseverance, we were able to overcome them and achieve successful results.
\par One of the key aspects of the implementation was the adoption of a microservice architecture using Docker. This allowed us to modularize the system and deploy different components in separate Docker containers. By leveraging Docker, we ensured scalability, flexibility, and portability of our solution. Each organization in the network, including peers and orderers, was hosted within its own Docker container, providing an isolated and consistent environment for running the Hyperledger Fabric nodes.
\par The Hyperledger Fabric framework served as the foundation for our blockchain implementation. It provided us with the necessary tools and protocols for building a secure and decentralized supply chain management system. By leveraging the features of Hyperledger Fabric, such as smart contracts and consensus mechanisms, we were able to create a transparent and tamper-proof ledger for tracking and verifying supply chain transactions.


% ----------------------------
% Brief outline of the study
% ----------------------------

\section{ Experimental Setup}
% \lipsum[22-25]

\subsection{Blockchain Network Setup}
% \lipsum[22-25]
\noindent In the initial step, we will set up the network infrastructure by installing and configuring the necessary dependencies such as Hyperledger Fabric, Docker, and Node.js. Then, we'll establish the network topology, consisting of three organizations (Org1, Org2, and Org3) and a single orderer node. Each organization will have its own cryptographic material generated using the Fabric CA, ensuring secure communication and authentication. 
\subsection{Channel configuration}
% \lipsum[22-25]
\noindent Next, we configured the channel, which involves defining the channel structure and membership. The channel acts as a private communication pathway between the organizations, providing confidentiality and restricted access to transactions and data. We utilize the configtxgen tool to generate the required channel artifacts, including the genesis block. This block serves as the starting point for the channel and contains the initial configuration information.

\subsection{Chaincode Development}
\noindent In this project, we developed a smart 
contract using the Fabric Contract API package and the Node.js SDK in Hyperledger Fabric. By leveraging the Fabric Contract API and Node.js SDK, we can simplify the contract development process and interact with the blockchain network seamlessly. In this section, we will define the contract logic, handle transactions, manage the ledger state, and interact with other network participants. The combination of the Fabric Contract API and Node.js SDK empowers us to build efficient and scalable smart contracts for decentralized applications.
\subsubsection{User Registration}
\begin{lstlisting}[language=Python , caption=User register]
input name, email, userType, address, password, profilePic
newUserID=generateUserID()
Store(newUserID, user)
incrementCounter('UserCounterNO')
return user
\end{lstlisting}
this is example on user register
\subsubsection{User Signin}
\begin{lstlisting}[language=Python , caption=User signin]
Input userId, password:
if userId andr password exist:
User <-Get(eId); 
if User.Password is equal to password:
return Success
\end{lstlisting}
The signIn function checks the provided user ID and password for emptiness. If either of them is empty, an error response is returned. The function then retrieves the user's data from the system's state using the provided user ID. If the user data is not found, an error response is returned indicating that the entity (user) cannot be found. The function compares the provided password with the password stored in the user's data. If they do not match, an error response is returned indicating that either the ID or password is wrong. If the password matches, relevant user data such as address, email, name, user ID, user type, and profile picture are extracted from the entityUser object. Finally, a success response is returned with status 200 and the extracted user data, indicating a successful sign-in.

\subsubsection{create product}
\begin{lstlisting}[language=Python , caption=Create product]
input name, mid, price, quantity:
User <- Get(mid)
if mid exist in the network:
If User.type == `manufacturer`:
productCounter <- Get(productCounter);
newProduct id <- generateProductId(productCounter);
timestampget <- Get(timestamp);. product = (name, mid, price, quantity);
Store(newProductId , Product , time );
incrementProductCounter(Product);
return success;       
\end{lstlisting}
The create product process includes several steps. First, the input parameters are validated to ensure their correctness. Then, the system verifies the manufacturer's existence and user type. The price is converted to a numeric format. A unique product ID is generated, and the current timestamp is obtained. An invoice is created, and the product object is constructed with relevant details. It is then stored in the system's database or ledger using the generated ID. The product counter is incremented to maintain uniqueness. Finally, a success response is returned to indicate the successful creation of the product.
\subsubsection{product listing}
\begin{lstlisting}[language=Python , caption=product listing]
input pid,uid,pprice,climate,soil type
if pid & uid exists in the network:
users <- Get(uid);
if user.type == `manufacture`:
product_price=price+raw_product_amount;
product(amount = price,production_data.climate=climate,production_data.soil_type=soil_type,status`Available`,product.availableFor=`exporter`)
if user.type ==`exporter`:
product_price=price+product.price;
product(amount=price,production_data.selectLogistic,status` Available` ,product.availableFor=`importer`)
if user.type ==`importer`:
product_price=price+ product.price;
product(amount=price,production_data.delivered=true,status`Processing`)
if user.type ==`logistic`:
product(amount=price,production_data.climate=climate,production_data.soil_type=soil_type,status`Available` ,product.availableFor=`retailer`)
if user.type==`retailer`:
product_price=price+ product.price;
product(amount=price,production_data.climate=climate,production_data.soil_type=soil_type,status`Available` ,product.availableFor=`consumer`)
updatedProductAsBytes= serializeToBytes(product);
return success== updatedProductAsBytes;
      
\end{lstlisting}
    This is example on product listing
\subsubsection{send to next stackholder}
\begin{lstlisting}[language=Python , caption=send to exporter]
input pId,eId:
if pId or eId exists in the network:
User <-Get(eId);
if user.type != `exporter`':
product=Get(eId);
txTimestamp=Get(timestamp);			 
invoice=[name:product.name,price:product.product.price,quantity:product.product.quantity]
product.exporter.id= eId;
product(availablefor==`importer`',status==`processing`',exporter status==`processing`');
return success;       
\end{lstlisting}
The ``sendToExporter" function facilitates the process of sending a product to an exporter. It takes the product ID and exporter ID as inputs. The function performs several verifications to ensure the validity of the inputs. It checks if the product and exporter exist in the system and verifies that the exporter has the appropriate user type. Once the verifications are successful, the function retrieves the product details and validates if the product has already been sent to an the stackholder. If the product is available, the function proceeds to create an invoice and update the necessary product and information. The stackholder ID, export date, product owner, available status, and invoice details are updated accordingly. The updated product object is serialized and stored back into the system. This process ensures that the product is successfully sent to the exporter and marked as ``Processing" for further export-related activities.
\subsubsection{invoice creation}
\begin{lstlisting}[language=Python , caption=invoice creation]
input product,from:
if product exists in the network:
productobj <-Get(productId);
sellerId <- productObj.product.owner
existingproduct <- Find(sellerId,ProductId);
if existingproduct is not NULL:
existingProduct.quantity += parseInt(product.quantity);
invoidedate <- GetTimestamp(ctx);
From seller Get invoicecounter,invoiceid,invoice;
save(newinvoiceId,invoice);
Push(sellerId,productId,invoicenumber);
return invoice;
       
\end{lstlisting}
The createInvoice takes in a product, which is parsed as a JSON object. It performs validations to ensure that at least one product is provided and that the manufacturer ID is provided. It then processes each product, grouping them by seller and calculating the total price for each seller's products. The function generates a unique invoice ID, creates an invoice object, and saves it to the ledger. Finally, it returns an array of invoices containing information such as the seller ID, product IDs, and invoice number.
\subsubsection{query}
\begin{lstlisting}[language=Python , caption=User query]
Input type, recordElement, recordValue
productbyType<-GET(Type)
if productByTypeExist:
assetCounter = parseInt(getCounter(assetType));
startKey = assetType + '1';
endKey = assetType + (assetCounter + 1);
resultsIterator=getStateByRange(startKey, endKey);
const buffer = [];
while (true) 
Record = resultIterator.next() 
if Record.product.type !== type):
if Record[recordElement].id === recordValue:
buffer.push({ Key, Record });
return { success: buffer  };   
\end{lstlisting}
The query function takes the input parameters include the type of product, the element to be searched , and its corresponding value . Then, the code attempts to retrieve products by their type using a GET request . If the products of the specified type exist, the code proceeds to fetch the asset counter for that type .Then the startKey and endKey are defined to set the range of assets to be retrieved. A resultsIterator is created to iterate through the state and obtain records falling within the specified range .A buffer is initialized to store the matching records . The code enters a loop , and within each iteration, it retrieves the next record . If the record's product type does not match the desired type , the loop continues to the next iteration. If the record's specified element  matches the provided value , the record is added to the buffer.Finally, the function returns the buffer containing the matching records , providing the result of the operation as a success along with the retrieved records.







\subsection{Machine Learning Model}
\noindent The developed solution presents a machine learning (ML) model for detecting the freshness of fruits using a ConvolutionalNeural Network (CNN) algorithm.The pseuodocode for our model is shown below \par
\begin{lstlisting}[language=Python , caption=ML model]
Mount Drive with Colab
Set Kaggle configuration and download dataset
Unzip the dataset
Access dataset in Drive and copy it to Colab
Prepare the data directories
Display sample images from the dataset
Preprocess the data using ImageDataGenerator
Build the model architecture
Compile the model with optimizer and loss function
Train the model on the training data
Evaluate the model's accuracy and loss
Run the model on uploaded images to make predictions
Save the trained model in different formats
Clean up the kernel
\end{lstlisting}
\noindent Then we use react component to create a page for image classification. It uses the TensorFlow.js library to load and run a the saved model for image classification. The component allows the user to classify images either from the webcam or from a local file. It also provides options to update the model, display predictions, and toggle the webcam.
\noindent 
%\lipsum[22-25]








