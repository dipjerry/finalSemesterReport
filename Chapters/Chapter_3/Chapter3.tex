% Chapter Template

\chapter{Feasibility Study}\doublespacing % Main chapter title

\label{Chapter3} % Change X to a consecutive number; for referencing this chapter elsewhere, use \ref{ChapterX}

\lhead{Chapter III. \emph{Feasibility Study}} % Change X to a consecutive number; this is for the header on each page - perhaps a shortened title

% ----------------------------
% Introduction
% ----------------------------

\section{Introduction}
We conducted a feasibility study to assess the viability and potential benefits of implementing a blockchain-based supply chain management system. Our goal was to determine whether adopting blockchain technology is feasible and advantageous for our supply chain management process. We evaluated various aspects, including technical, economic, operational, and organizational factors, to make an informed decision. 


% ----------------------------
% Brief outline of the study
% ----------------------------

\subsection{Technical Feasibility}
\noindent We evaluated the technical capabilities and requirements of implementing a blockchain solution in our existing supply chain infrastructure. We assessed the compatibility of the blockchain technology with our current IT systems and software. We also considered the scalability, security, and performance aspects of the blockchain network to ensure it can handle our supply chain transactions.


% ----------------------------
% Problem Statement
% ----------------------------

\subsection{Economic Feasibility}
\noindent We conducted a comprehensive cost-benefit analysis to determine the financial viability of implementing a blockchain-based supply chain management system. We assessed the initial setup costs, including hardware, software, and infrastructure requirements. Additionally, we analyzed the potential cost savings and efficiency gains that could be achieved through enhanced transparency, traceability, and reduced intermediaries in our supply chain process. Based on this analysis, we determined the return on investment (ROI) and payback period for implementing the blockchain solution. 

% \begin{itemize}

%   \item Nullam bibendum, turpis vitae tristique gravida, quam sapien tempor lectus, quis pretium tellus purus ac quam.
  
%   \item Cibendum, turpis vitae tristique gravida, quam sapien tempor lectus, quis pretium tellus purus ac quam.
  
%   \item Turpis vitae tristique gravida, quam sapien tempor lectus, quis pretium tellus purus ac quam. 
  
% \end{itemize}

% % ----------------------------
% % Objective of the Study
% % ----------------------------

% \section{Objective of the Study}
%  Morbi non felis ac libero vulputate fringilla. Mauris
% libero eros, lacinia non, sodales quis, dapibus porttitor, pede. Class aptent taciti sociosqu ad litora torquent per conubia nostra, per inceptos hymenaeos:

% \begin{itemize}

%   \item Sodales quis, dapibus porttitor, pede. Class aptent taciti sociosqu ad litora torquent per conubia nostra, per inceptos hymenaeos.
  
%   \item Lacinia non, sodales quis, dapibus porttitor, pede. Class aptent taciti sociosqu ad litora torquent .
  
% \end{itemize}


% ----------------------------
% Methodology
% ----------------------------

\subsection{Operational Feasibility}
\noindent We assessed the impact of implementing a blockchain system on our day-to-day supply chain operations. We identified the potential operational benefits, such as real-time tracking, improved inventory management, and streamlined logistics. We also considered the training and skill requirements for our employees to adapt to the new technology and ensure a smooth adoption and implementation process. Throughout this assessment, we analyzed any potential disruptions or challenges that may arise during the transition phase and developed strategies to mitigate risks.

% \section{Conclusion}
\noindent Based on our feasibility study, we concluded that implementing a blockchain-based supply chain management system is feasible and beneficial for a organization. The study provided valuable insights into the technical, economic, operational, and organizational aspects, enabling us to make informed decisions and plan for the adoption of blockchain technology in our supply chain management processes.







